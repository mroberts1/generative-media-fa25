% Options for packages loaded elsewhere
% Options for packages loaded elsewhere
\PassOptionsToPackage{unicode}{hyperref}
\PassOptionsToPackage{hyphens}{url}
\PassOptionsToPackage{dvipsnames,svgnames,x11names}{xcolor}
%
\documentclass[
  letterpaper,
  DIV=11,
  numbers=noendperiod]{scrartcl}
\usepackage{xcolor}
\usepackage{amsmath,amssymb}
\setcounter{secnumdepth}{-\maxdimen} % remove section numbering
\usepackage{iftex}
\ifPDFTeX
  \usepackage[T1]{fontenc}
  \usepackage[utf8]{inputenc}
  \usepackage{textcomp} % provide euro and other symbols
\else % if luatex or xetex
  \usepackage{unicode-math} % this also loads fontspec
  \defaultfontfeatures{Scale=MatchLowercase}
  \defaultfontfeatures[\rmfamily]{Ligatures=TeX,Scale=1}
\fi
\usepackage{lmodern}
\ifPDFTeX\else
  % xetex/luatex font selection
\fi
% Use upquote if available, for straight quotes in verbatim environments
\IfFileExists{upquote.sty}{\usepackage{upquote}}{}
\IfFileExists{microtype.sty}{% use microtype if available
  \usepackage[]{microtype}
  \UseMicrotypeSet[protrusion]{basicmath} % disable protrusion for tt fonts
}{}
\makeatletter
\@ifundefined{KOMAClassName}{% if non-KOMA class
  \IfFileExists{parskip.sty}{%
    \usepackage{parskip}
  }{% else
    \setlength{\parindent}{0pt}
    \setlength{\parskip}{6pt plus 2pt minus 1pt}}
}{% if KOMA class
  \KOMAoptions{parskip=half}}
\makeatother
% Make \paragraph and \subparagraph free-standing
\makeatletter
\ifx\paragraph\undefined\else
  \let\oldparagraph\paragraph
  \renewcommand{\paragraph}{
    \@ifstar
      \xxxParagraphStar
      \xxxParagraphNoStar
  }
  \newcommand{\xxxParagraphStar}[1]{\oldparagraph*{#1}\mbox{}}
  \newcommand{\xxxParagraphNoStar}[1]{\oldparagraph{#1}\mbox{}}
\fi
\ifx\subparagraph\undefined\else
  \let\oldsubparagraph\subparagraph
  \renewcommand{\subparagraph}{
    \@ifstar
      \xxxSubParagraphStar
      \xxxSubParagraphNoStar
  }
  \newcommand{\xxxSubParagraphStar}[1]{\oldsubparagraph*{#1}\mbox{}}
  \newcommand{\xxxSubParagraphNoStar}[1]{\oldsubparagraph{#1}\mbox{}}
\fi
\makeatother

\usepackage{color}
\usepackage{fancyvrb}
\newcommand{\VerbBar}{|}
\newcommand{\VERB}{\Verb[commandchars=\\\{\}]}
\DefineVerbatimEnvironment{Highlighting}{Verbatim}{commandchars=\\\{\}}
% Add ',fontsize=\small' for more characters per line
\usepackage{framed}
\definecolor{shadecolor}{RGB}{241,243,245}
\newenvironment{Shaded}{\begin{snugshade}}{\end{snugshade}}
\newcommand{\AlertTok}[1]{\textcolor[rgb]{0.68,0.00,0.00}{#1}}
\newcommand{\AnnotationTok}[1]{\textcolor[rgb]{0.37,0.37,0.37}{#1}}
\newcommand{\AttributeTok}[1]{\textcolor[rgb]{0.40,0.45,0.13}{#1}}
\newcommand{\BaseNTok}[1]{\textcolor[rgb]{0.68,0.00,0.00}{#1}}
\newcommand{\BuiltInTok}[1]{\textcolor[rgb]{0.00,0.23,0.31}{#1}}
\newcommand{\CharTok}[1]{\textcolor[rgb]{0.13,0.47,0.30}{#1}}
\newcommand{\CommentTok}[1]{\textcolor[rgb]{0.37,0.37,0.37}{#1}}
\newcommand{\CommentVarTok}[1]{\textcolor[rgb]{0.37,0.37,0.37}{\textit{#1}}}
\newcommand{\ConstantTok}[1]{\textcolor[rgb]{0.56,0.35,0.01}{#1}}
\newcommand{\ControlFlowTok}[1]{\textcolor[rgb]{0.00,0.23,0.31}{\textbf{#1}}}
\newcommand{\DataTypeTok}[1]{\textcolor[rgb]{0.68,0.00,0.00}{#1}}
\newcommand{\DecValTok}[1]{\textcolor[rgb]{0.68,0.00,0.00}{#1}}
\newcommand{\DocumentationTok}[1]{\textcolor[rgb]{0.37,0.37,0.37}{\textit{#1}}}
\newcommand{\ErrorTok}[1]{\textcolor[rgb]{0.68,0.00,0.00}{#1}}
\newcommand{\ExtensionTok}[1]{\textcolor[rgb]{0.00,0.23,0.31}{#1}}
\newcommand{\FloatTok}[1]{\textcolor[rgb]{0.68,0.00,0.00}{#1}}
\newcommand{\FunctionTok}[1]{\textcolor[rgb]{0.28,0.35,0.67}{#1}}
\newcommand{\ImportTok}[1]{\textcolor[rgb]{0.00,0.46,0.62}{#1}}
\newcommand{\InformationTok}[1]{\textcolor[rgb]{0.37,0.37,0.37}{#1}}
\newcommand{\KeywordTok}[1]{\textcolor[rgb]{0.00,0.23,0.31}{\textbf{#1}}}
\newcommand{\NormalTok}[1]{\textcolor[rgb]{0.00,0.23,0.31}{#1}}
\newcommand{\OperatorTok}[1]{\textcolor[rgb]{0.37,0.37,0.37}{#1}}
\newcommand{\OtherTok}[1]{\textcolor[rgb]{0.00,0.23,0.31}{#1}}
\newcommand{\PreprocessorTok}[1]{\textcolor[rgb]{0.68,0.00,0.00}{#1}}
\newcommand{\RegionMarkerTok}[1]{\textcolor[rgb]{0.00,0.23,0.31}{#1}}
\newcommand{\SpecialCharTok}[1]{\textcolor[rgb]{0.37,0.37,0.37}{#1}}
\newcommand{\SpecialStringTok}[1]{\textcolor[rgb]{0.13,0.47,0.30}{#1}}
\newcommand{\StringTok}[1]{\textcolor[rgb]{0.13,0.47,0.30}{#1}}
\newcommand{\VariableTok}[1]{\textcolor[rgb]{0.07,0.07,0.07}{#1}}
\newcommand{\VerbatimStringTok}[1]{\textcolor[rgb]{0.13,0.47,0.30}{#1}}
\newcommand{\WarningTok}[1]{\textcolor[rgb]{0.37,0.37,0.37}{\textit{#1}}}

\usepackage{longtable,booktabs,array}
\usepackage{calc} % for calculating minipage widths
% Correct order of tables after \paragraph or \subparagraph
\usepackage{etoolbox}
\makeatletter
\patchcmd\longtable{\par}{\if@noskipsec\mbox{}\fi\par}{}{}
\makeatother
% Allow footnotes in longtable head/foot
\IfFileExists{footnotehyper.sty}{\usepackage{footnotehyper}}{\usepackage{footnote}}
\makesavenoteenv{longtable}
\usepackage{graphicx}
\makeatletter
\newsavebox\pandoc@box
\newcommand*\pandocbounded[1]{% scales image to fit in text height/width
  \sbox\pandoc@box{#1}%
  \Gscale@div\@tempa{\textheight}{\dimexpr\ht\pandoc@box+\dp\pandoc@box\relax}%
  \Gscale@div\@tempb{\linewidth}{\wd\pandoc@box}%
  \ifdim\@tempb\p@<\@tempa\p@\let\@tempa\@tempb\fi% select the smaller of both
  \ifdim\@tempa\p@<\p@\scalebox{\@tempa}{\usebox\pandoc@box}%
  \else\usebox{\pandoc@box}%
  \fi%
}
% Set default figure placement to htbp
\def\fps@figure{htbp}
\makeatother


% definitions for citeproc citations
\NewDocumentCommand\citeproctext{}{}
\NewDocumentCommand\citeproc{mm}{%
  \begingroup\def\citeproctext{#2}\cite{#1}\endgroup}
\makeatletter
 % allow citations to break across lines
 \let\@cite@ofmt\@firstofone
 % avoid brackets around text for \cite:
 \def\@biblabel#1{}
 \def\@cite#1#2{{#1\if@tempswa , #2\fi}}
\makeatother
\newlength{\cslhangindent}
\setlength{\cslhangindent}{1.5em}
\newlength{\csllabelwidth}
\setlength{\csllabelwidth}{3em}
\newenvironment{CSLReferences}[2] % #1 hanging-indent, #2 entry-spacing
 {\begin{list}{}{%
  \setlength{\itemindent}{0pt}
  \setlength{\leftmargin}{0pt}
  \setlength{\parsep}{0pt}
  % turn on hanging indent if param 1 is 1
  \ifodd #1
   \setlength{\leftmargin}{\cslhangindent}
   \setlength{\itemindent}{-1\cslhangindent}
  \fi
  % set entry spacing
  \setlength{\itemsep}{#2\baselineskip}}}
 {\end{list}}
\usepackage{calc}
\newcommand{\CSLBlock}[1]{\hfill\break\parbox[t]{\linewidth}{\strut\ignorespaces#1\strut}}
\newcommand{\CSLLeftMargin}[1]{\parbox[t]{\csllabelwidth}{\strut#1\strut}}
\newcommand{\CSLRightInline}[1]{\parbox[t]{\linewidth - \csllabelwidth}{\strut#1\strut}}
\newcommand{\CSLIndent}[1]{\hspace{\cslhangindent}#1}



\setlength{\emergencystretch}{3em} % prevent overfull lines

\providecommand{\tightlist}{%
  \setlength{\itemsep}{0pt}\setlength{\parskip}{0pt}}



 


% load packages
\usepackage{geometry}
\usepackage{xcolor}
\usepackage{eso-pic}
\usepackage{fancyhdr}
\usepackage{sectsty}
\usepackage{fontspec}
\usepackage{titlesec}

%% Set page size with a wider right margin
\geometry{a4paper, total={170mm,257mm}, left=20mm, top=20mm, bottom=20mm, right=50mm}

%% Let's define some colours
\definecolor{light}{HTML}{E6E6FA}
\definecolor{highlight}{HTML}{800080}
\definecolor{dark}{HTML}{330033}

%% Let's add the border on the right hand side 
% \AddToShipoutPicture{% 
%     \AtPageLowerLeft{% 
%         \put(\LenToUnit{\dimexpr\paperwidth-3cm},0){% 
%             \color{light}\rule{3cm}{\LenToUnit\paperheight}%
%           }%
%      }%
%      % logo
%     \AtPageLowerLeft{% start the bar at the bottom right of the page
%         \put(\LenToUnit{\dimexpr\paperwidth-2.25cm},27.2cm){% move it to the top right
%             \color{light}\includegraphics[width=1.5cm]{_extensions/nrennie/PrettyPDF/logo.png}
%           }%
%      }%
% }

%% Style the page number
\fancypagestyle{mystyle}{
  \fancyhf{}
  \renewcommand\headrulewidth{0pt}
  \fancyfoot[R]{\thepage}
  \fancyfootoffset{3.5cm}
}
\setlength{\footskip}{20pt}

%% style the chapter/section fonts
\chapterfont{\color{dark}\fontsize{20}{16.8}\selectfont}
\sectionfont{\color{dark}\fontsize{20}{16.8}\selectfont}
\subsectionfont{\color{dark}\fontsize{14}{16.8}\selectfont}
\titleformat{\subsection}
  {\sffamily\Large\bfseries}{\thesection}{1em}{}[{\titlerule[0.8pt]}]
  
% left align title
\makeatletter
\renewcommand{\maketitle}{\bgroup\setlength{\parindent}{0pt}
\begin{flushleft}
  {\sffamily\huge\textbf{\MakeUppercase{\@title}}} \vspace{0.3cm} \newline
  {\Large {\@subtitle}} \newline
  \@author
\end{flushleft}\egroup
}
\makeatother

%% Use some custom fonts
\setsansfont{Ubuntu}[
    Path=_extensions/nrennie/PrettyPDF/Ubuntu/,
    Scale=0.9,
    Extension = .ttf,
    UprightFont=*-Regular,
    BoldFont=*-Bold,
    ItalicFont=*-Italic,
    ]

\setmainfont{Ubuntu}[
    Path=_extensions/nrennie/PrettyPDF/Ubuntu/,
    Scale=0.9,
    Extension = .ttf,
    UprightFont=*-Regular,
    BoldFont=*-Bold,
    ItalicFont=*-Italic,
    ]
\KOMAoption{captions}{tableheading}
\makeatletter
\@ifpackageloaded{tcolorbox}{}{\usepackage[skins,breakable]{tcolorbox}}
\@ifpackageloaded{fontawesome5}{}{\usepackage{fontawesome5}}
\definecolor{quarto-callout-color}{HTML}{909090}
\definecolor{quarto-callout-note-color}{HTML}{0758E5}
\definecolor{quarto-callout-important-color}{HTML}{CC1914}
\definecolor{quarto-callout-warning-color}{HTML}{EB9113}
\definecolor{quarto-callout-tip-color}{HTML}{00A047}
\definecolor{quarto-callout-caution-color}{HTML}{FC5300}
\definecolor{quarto-callout-color-frame}{HTML}{acacac}
\definecolor{quarto-callout-note-color-frame}{HTML}{4582ec}
\definecolor{quarto-callout-important-color-frame}{HTML}{d9534f}
\definecolor{quarto-callout-warning-color-frame}{HTML}{f0ad4e}
\definecolor{quarto-callout-tip-color-frame}{HTML}{02b875}
\definecolor{quarto-callout-caution-color-frame}{HTML}{fd7e14}
\makeatother
\makeatletter
\@ifpackageloaded{caption}{}{\usepackage{caption}}
\AtBeginDocument{%
\ifdefined\contentsname
  \renewcommand*\contentsname{Table of contents}
\else
  \newcommand\contentsname{Table of contents}
\fi
\ifdefined\listfigurename
  \renewcommand*\listfigurename{List of Figures}
\else
  \newcommand\listfigurename{List of Figures}
\fi
\ifdefined\listtablename
  \renewcommand*\listtablename{List of Tables}
\else
  \newcommand\listtablename{List of Tables}
\fi
\ifdefined\figurename
  \renewcommand*\figurename{Figure}
\else
  \newcommand\figurename{Figure}
\fi
\ifdefined\tablename
  \renewcommand*\tablename{Table}
\else
  \newcommand\tablename{Table}
\fi
}
\@ifpackageloaded{float}{}{\usepackage{float}}
\floatstyle{ruled}
\@ifundefined{c@chapter}{\newfloat{codelisting}{h}{lop}}{\newfloat{codelisting}{h}{lop}[chapter]}
\floatname{codelisting}{Listing}
\newcommand*\listoflistings{\listof{codelisting}{List of Listings}}
\makeatother
\makeatletter
\makeatother
\makeatletter
\@ifpackageloaded{caption}{}{\usepackage{caption}}
\@ifpackageloaded{subcaption}{}{\usepackage{subcaption}}
\makeatother
\makeatletter
\@ifpackageloaded{tcolorbox}{}{\usepackage[skins,breakable]{tcolorbox}}
\makeatother
\makeatletter
\@ifundefined{shadecolor}{\definecolor{shadecolor}{rgb}{.97, .97, .97}}{}
\makeatother
\makeatletter
\@ifundefined{codebgcolor}{\definecolor{codebgcolor}{named}{light}}{}
\makeatother
\makeatletter
\ifdefined\Shaded\renewenvironment{Shaded}{\begin{tcolorbox}[colback={codebgcolor}, boxrule=0pt, enhanced, breakable, frame hidden, sharp corners]}{\end{tcolorbox}}\fi
\makeatother
\makeatletter
\@ifpackageloaded{fontawesome5}{}{\usepackage{fontawesome5}}
\makeatother
\usepackage{bookmark}
\IfFileExists{xurl.sty}{\usepackage{xurl}}{} % add URL line breaks if available
\urlstyle{same}
\hypersetup{
  pdftitle={VM402-09 Generative Media},
  colorlinks=true,
  linkcolor={highlight},
  filecolor={Maroon},
  citecolor={Blue},
  urlcolor={highlight},
  pdfcreator={LaTeX via pandoc}}


\title{VM402-09 Generative Media}
\usepackage{etoolbox}
\makeatletter
\providecommand{\subtitle}[1]{% add subtitle to \maketitle
  \apptocmd{\@title}{\par {\large #1 \par}}{}{}
}
\makeatother
\subtitle{History, Theory, Practice}
\author{}
\date{}
\begin{document}
\maketitle

\pagestyle{mystyle}


\begin{figure}

\begin{minipage}{0.61\linewidth}
\href{https://www.instagram.com/p/C4fu8cmMD0G/?img_index=1}{\pandocbounded{\includegraphics[keepaspectratio]{gif/outlanders-brain-magazine.gif}}}\end{minipage}%
%
\begin{minipage}{0.03\linewidth}
\href{https://emerson.edu/academics/academic-departments/visual-media-arts}{Department
of Visual \& Media Arts}\\
\href{https://emerson.edu/}{Emerson College}\\
Summer 2025\\
Wed 2 July---Wed 20 August (7 weeks) Online/asynchronous\\
\href{http://mroberts.emerson.build/}{Dr.~Martin Roberts}
\href{mailto:martin_roberts@emerson.edu}{\faIcon{envelope}} \textbar{}
\href{https://merveilles.town/@dokoissho}{\faIcon{mastodon}} \textbar{}
\href{https://github.com/mroberts1/}{\faIcon{github}} \textbar{}
\href{https://twitter.com/mroberts_vma}{\faIcon{twitter}}\\
\end{minipage}%

\end{figure}%

Image credit: \href{https://outlandersdesign.com}{Outlanders Design}

\subsection{Interactive Boids
Simulation}\label{interactive-boids-simulation}

\emph{Demonstrating emergent behavior and algorithmic art - a perfect
example of generative media in action.}

\begin{Shaded}
\begin{Highlighting}[]
\NormalTok{import \{canvas\} from "@harrystevens/p5{-}js{-}boids"}
\NormalTok{canvas}
\end{Highlighting}
\end{Shaded}

\begin{tcolorbox}[enhanced jigsaw, title=\textcolor{quarto-callout-note-color}{\faInfo}\hspace{0.5em}{Note}, colframe=quarto-callout-note-color-frame, coltitle=black, colback=white, leftrule=.75mm, opacityback=0, left=2mm, titlerule=0mm, toprule=.15mm, breakable, rightrule=.15mm, bottomtitle=1mm, toptitle=1mm, bottomrule=.15mm, colbacktitle=quarto-callout-note-color!10!white, arc=.35mm, opacitybacktitle=0.6]

\textbf{About this simulation:} This interactive visualization
demonstrates Craig Reynolds' ``boids'' algorithm, which simulates
flocking behavior through simple rules: separation (avoid crowding
neighbors), alignment (steer towards average heading of neighbors), and
cohesion (steer towards average position of neighbors). The emergent
complexity from these simple rules exemplifies key concepts in
generative media.

\end{tcolorbox}

\begin{center}\rule{0.5\linewidth}{0.5pt}\end{center}

Communication plan:

The best way to reach me is by emailing me (martin\_roberts@emerson.edu)
and I will respond within 12 to 24 hours upon receiving your email
message.

I am also available for zoom meetings. Please email me with your meeting
requests and available windows.

Zoom Office Hours: by appointment

https://emerson.zoom.us/j/96415591363 (Passcode: 923351)

\subsection{Description}\label{description}

This online, asynchronous course examines approaches to the analysis and
criticism of contemporary media. You will learn how to become critical
of the media messages you use \emph{and} create, and how to think and
write with agency.

There are two components of this course:

\begin{enumerate}
\def\labelenumi{\arabic{enumi}.}
\item
  A survey of such critical approaches/methods as semiotics
  (\citeproc{ref-eco1986semiological}{Eco 1986};
  \citeproc{ref-barthes1972mythologies}{Barthes 1972}), psychoanalysis,
  genre studies, feminist theories
  (\citeproc{ref-negra2007introduction}{Negra and Tasker 2007}),
  cultural studies (\citeproc{ref-hall1980encoding}{Hall 1980}), and
  postmodernism (\citeproc{ref-baudrillard1994simulacra}{Baudrillard
  1994}; \citeproc{ref-jameson1991postmodernism}{Jameson 1991});
\item
  The application of these approaches to examining and critiquing
  various media forms and contents. We will spend a major part of this
  semester building up theoretical/methodological foundation while
  devoting the remaining to the critique and analysis of specific media
  texts and contexts.
\end{enumerate}

Through extensive reading, writing, class discussion and screening, you
will begin to develop the ability to analyze and critique popular media
messages and images (\citeproc{ref-mulvey1975visual}{Mulvey 1975}), as
well as develop a critical worldview informed by contemporary media
theory (\citeproc{ref-fisher2009capitalist}{Fisher 2009}).

\subsection{Learning Objectives}\label{learning-objectives}

By the end of the course, students should be able to:

\begin{enumerate}
\def\labelenumi{\arabic{enumi}.}
\tightlist
\item
  Identify a variety of critical and theoretical approaches to the media
\item
  Develop and refine critical reading strategies, both in terms of the
  media and writings about the media
\item
  Cultivate a suitable academic vocabulary and the ability to write,
  think and create from a critical perspective
\item
  Appreciate an ethically-based and de-centered global perspective on
  media form and function, sensitive to the work of minority critics and
  theorists under-represented in the canons of media criticism
\item
  Apply theory to practice, and engage with contemporary media
  production and consumption as critically aware media artists
\end{enumerate}

\subsection{Required Texts}\label{required-texts}

Mark Fisher, \emph{Capitalist Realism: Is There No Alternative?}
(Winchester, UK, and Washington, USA: Zero Books, 2009).

Allissa V. Richardson, \emph{Bearing Witness While Black: African
Americans, Smartphones, and the New Protest \#Journalism} (Oxford:
Oxford University Press, 2020).

Supplementary readings, available on Canvas (see bibliography).

\subsection{Course Structure and
Requirements}\label{course-structure-and-requirements}

\textbf{Schedule}\\
The course begins on Wednesday 2 July and ends on Tuesday 19 August.
Each week of the course accordingly begins on a \textbf{Wednesday} and
ends on the following \textbf{Tuesday}.

\textbf{Lectures, Discussion Forums, and Weekly Response Posts}\\
Each week, I will post a lecture, either in textual or video format,
usually by \textbf{Friday} of the week in question.

In the lecture, I will assume that you have done the reading assignments
for the week, so make sure to complete them before reading (or watching)
the lecture.

The lecture will typically introduce some relevant examples relating to
the readings that we can use as a starting point for the week's
discussion. However, you are also very welcome to suggest examples of
your own that may come to mind after reading the course materials and
the lecture.

Each week, your initial Response post is due no later than the last day
of the week in question (i.e.~Tuesday in this case). You are then
required to post at least 2 follow-up responses to any other group
members' posts (or my own) during the subsequent week. The Discussion
cycle for each weekly topic therefore extends over two weeks:

\begin{itemize}
\tightlist
\item
  By the end of the first week (Tuesday): initial response post
\item
  By the end of second week (the following Tuesday): min. 2 follow-ups
  to other people's posts
\end{itemize}

Don't sweat too much about meeting exact requirements for post length
and frequency! The purpose is for us to have a lively, productive
discussion of each week's readings so keep that as your main goal.
Forums will also stay open beyond the biweekly cycle so you can catch up
with posts later if you have time.

\textbf{Assigned Readings}\\
Plan on reading every assignment as listed on the course calendar. Since
the reading will both inform and enhance material covered during class,
please complete the reading assignments \emph{prior to} watching the
lecture videos.

\textbf{Attendance and Participation}\\
Full participation in class discussions and timely submission of all
assignments constitutes evidence of class attendance. Students are
expected to allocate sufficient time to complete all the requirements
for each module: reading, viewing, responding to discussion questions,
submitting replies to classmates, and completing other assignments on
time. Failure to keep up with the pace of the course may result in lower
grades or the failure of the class.

\textbf{Discussion Forums and Weekly Response Posts}\\
Each week you are required to post a response of approximately 100-250
words to the course readings and/or the weekly lecture. The purpose of
this assignment is for you to reflect on lectures, reading assignments
or media clips and to pose questions. Along with your initial post, you
are also required to respond to at least two (2) of your peers'
comments. Response posts are usually due by the end of the week in
question: in this case, since the course began on a Wednesday, posts
will be due by the Tuesday of the following week.

\textbf{Individual Papers}\\
Each of you will write two analytical papers, which are due on June 9
and July 7 respectively. For the second paper assignment, you have the
option to do a creative project (instead of a paper) using either BCS or
Postmodernism concepts. Detailed instructions of the assignments are
posted on Canvas. Feel free to consult with me if you have questions
about an assignment or think you need extra help.

\textbf{Response Papers}\\
You will write two response papers responding to assigned films (see
syllabus) in the contexts of the assigned readings. Each response paper
should be approximately 750-1,000 words in length (3-4 pages,
double-spaced). Your grade depends on how well you analyze and critique
the film and how well you incorporate the readings into your
critique/analysis.

The grading scale for papers is as follows:

\begin{longtable}[]{@{}
  >{\raggedright\arraybackslash}p{(\linewidth - 2\tabcolsep) * \real{0.5000}}
  >{\raggedright\arraybackslash}p{(\linewidth - 2\tabcolsep) * \real{0.5000}}@{}}
\toprule\noalign{}
\endhead
\bottomrule\noalign{}
\endlastfoot
A & This is excellent scholarly work that could be published in a
student film or media studies journal. It displays good \emph{structure}
(i.e., a strong original thesis statement and a logical development of
points to follow), \emph{content} (sound analysis and good research when
applicable), and \emph{presentational skills} (i.e., excellence in
verbal, logical, and grammatical expression, as well as care in
mechanical details such as spelling, typing, bibliography and footnote
preparation, etc.). It also follows style and formatting rules.
(A=95). \\
B & This is good scholarly work that could be published in a student
journal with some adjustments. It has minor problems in a couple of the
aforementioned areas. (B=85) \\
C & This work exhibits problems in several of the aforementioned areas
or a major problem in one. (C=75) \\
D & This work exhibits little understanding of the subject matter or
paper requirements. It fails to follow rudimentary formatting and
displays sloppiness in spelling, grammar and sourcing. (D=65) \\
F & This work exhibits no understanding of the subject matter or paper
requirements. It fails to follow rudimentary formatting and displays
sloppiness in spelling, grammar and sourcing. (F=55) \\
\end{longtable}

\subsection{Academic Honesty}\label{academic-honesty}

Do not plagiarize. Plagiarism is using someone else's work without
giving them fair credit. Plagiarism and cheating will result in an ``F''
for the assignment.

\subsection{Accommodations for Students with
Disabilities}\label{accommodations-for-students-with-disabilities}

Emerson is committed to providing equal access and support to all
qualified students through the provision of reasonable accommodations,
so that each student may fully participate in the Emerson experience.
Student Accessibility Services (SAS) staff will be working remotely for
the fall of 2020. If you have a disability that may require
accommodations, please contact them at
\href{mailto:SAS@emerson.edu}{\emph{SAS@emerson.edu}} or at
(617)824-8592 to make an appointment with an SAS staff member.

Students are encouraged to contact SAS early in the semester. Please be
aware that accommodations are not applied retroactively.

\subsection{Plagiarism Statement}\label{plagiarism-statement}

It is the responsibility of all Emerson students to know and adhere to
the College's policy on plagiarism, which can be found at:
\href{http://www.emerson.edu/policy/plagiarism}{\emph{http://www.emerson.edu/policy/plagiarism}}*.
If you have any question concerning the Emerson plagiarism policy or
about documentation of sources in work you produce in this course, speak
to your instructor.

\subsection{Diversity Statement}\label{diversity-statement}

Every student in this class will be honored and respected as an
individual with distinct experiences, talents, and backgrounds. Students
will be treated fairly regardless of race, religion, sexual orientation,
gender identification, disability, socio-economic status, or national
identity. Issues of diversity may be a part of class discussion,
assigned material, and projects. The instructor will make every effort
to ensure that an inclusive environment exists for all students. If you
have any concerns or suggestions for improving the classroom climate,
please do not hesitate to speak with the course instructor or to contact
the Office of Diversity and Inclusion at 617-824-8528 or by email at
\href{mailto:diversity*inclusion@emerson.edu}{\emph{diversity}inclusion@emerson.edu*}

\subsection{Grading}\label{grading}

Paper \#1 20\%

Paper \#2/Creative Project 20\%

Discussion Posts 30\%

Response papers (2) 30\%

\subsection{Class Schedule}\label{class-schedule}

\emph{Week 1: Wed 2 July---Tues 8 July}

\textbf{Introduction: Sites of Struggle}

Introduction to the course (Contemporary Criticism + Discourse);
Semiotics (signs, denotation, connotation, paradigm \& syntagm)
(\citeproc{ref-eco1986semiological}{Eco 1986})

To read:

\begin{itemize}
\tightlist
\item
  David Welch, ``The Conquest of the Masses''
\item
  Umberto Eco, ``Towards a Semiological Guerrilla Warfare''
\end{itemize}

\emph{Week 2: Wed 9 July---Tues 15 July}

\textbf{Obey: Ideology and the Orwellian}

Exploring how ideology shapes media representation and consumption,
drawing on Althusser's concept of ideological state apparatuses
(\citeproc{ref-althusser1971ideology}{Althusser 1971}) and Fisher's
analysis of capitalist realism
(\citeproc{ref-fisher2009capitalist}{Fisher 2009}).

To read:

\begin{itemize}
\tightlist
\item
  Louis Althusser, ``Ideology and Ideological State Apparatuses''
\item
  Mark Fisher, ``It's Easier to Imagine the End of the World Than the
  End of Capitalism''
\item
  Mark Fisher, ``Lecture 1: What Is Postcapitalist Desire?''
\end{itemize}

To watch (if you haven't already!): \emph{Squid Game}

\emph{Week 3: Wed 16 July---Tues 22 July}

\textbf{Simulation and Pastiche: Postmodernism}

Examining postmodern media culture through the lens of simulation theory
and performativity, using contemporary Instagram art as a case study
(\citeproc{ref-ulman2014excellences}{Ulman 2014}).

To read:

\begin{itemize}
\tightlist
\item
  Amalia Ulman, ``Excellences \& Perfections''
  (\href{http://amaliaulman.eu/}{website};
  \href{https://www.instagram.com/amaliaulman/}{Instagram})
\item
  Alastair Sooke, ``Is This The First Instagram Masterpiece?''
\item
  ``Amalia Ulman: Meme Come True''
\item
  Emilie Friedlander, ``Social Anxiety: Why Amalia Ulman's Fake
  `Middlebrow' Instagram Is No Different From Yours''
\end{itemize}

To watch:

\begin{itemize}
\tightlist
\item
  \emph{El Planeta}
\item
  \emph{Inventing Anna}, Episode 6: ``Friends in Low Places''
\end{itemize}

\textbf{DEADLINE: Tuesday 22 July: Paper 1}

\emph{Week 4: Wed 23 July---Tues 29 July}

\textbf{Feminism/Postfeminism}

Analyzing the shift from feminist to postfeminist media culture
(\citeproc{ref-negra2007introduction}{Negra and Tasker 2007}) and its
implications for contemporary media representation.

\begin{itemize}
\tightlist
\item
  Diane Negra and Yvonne Tasker, ``Introduction: Feminist Politics and
  Postfeminist Culture''
  \href{pdf/interrogating-postfeminism-intro.pdf}{PDF}
\item
  Martin Roberts, ``The Fashion Police: Governing the Self in \emph{What
  Not To Wear}'' \href{pdf/fashion-police.pdf}{PDF}
\end{itemize}

\emph{Week 5: Wed 30 July---Tues 5 August}

\textbf{Witnessing}

Examining the role of digital media in documenting social justice and
the concept of ``bearing witness'' in contemporary protest movements
(\citeproc{ref-richardson2020bearing}{Richardson 2020}).

Allissa V. Richardson, \emph{Bearing Witness While Black}:

\begin{itemize}
\tightlist
\item
  Chapter 1: ``Looking As Rebellion: The Concept of Black Witnessing''
  \href{pdf/richardson-bearing-witness-ch1.pdf}{PDF}
\item
  Chapter 3: ``The New Protest \#Journalism: Black Witnessing as
  Counternarrative'' \href{richardson-bearing-witness-ch3.pdf}{PDF}
\end{itemize}

\textbf{DEADLINE: Wednesday 6 August: Response Paper 1 (}El
Planeta\textbf{)}

\emph{Week 6: Wed 6 August---Tues 12 August}

\textbf{Tribes: Subcultures, Lifestyles, Aesthetics}

Investigating how internet aesthetics function as forms of identity
construction and community building
(\citeproc{ref-giolo2024aesthetics}{Giolo and Berghman 2024}).

\begin{itemize}
\tightlist
\item
  Guilherme Giolo and Michaël Berghman, ``The Aesthetics of the Self:
  The Meaning-Making of Internet Aesthetics''
  \href{pdf/giolo-berghman-aesthetics.pdf}{PDF}
\item
  \href{https://aesthetics.fandom.com/wiki/Aesthetics_Wiki}{Aesthetics
  Wiki} (read all articles in the section called ``What Are
  Aesthetics?'' and explore the site)
\end{itemize}

\emph{Week 7: Wed 13 August---Tues 19 August}

\textbf{Fandom}

Exploring participatory culture and the social dynamics of digital fan
communities (\citeproc{ref-baym2018music}{Baym et al. 2018};
\citeproc{ref-jenkins2012textual}{Jenkins 2012}).

\begin{itemize}
\tightlist
\item
  Nancy K. Baym, ``\href{pdf/baym-social-media-society.pdf}{Social Media
  and the Struggle for Society}''\\
\item
  Nancy K. Baym, Daniel Cavicchi, and Norma Coates,
  ``\href{pdf/baym-music-fandom.pdf}{Music Fandom in the Digital Age: A
  Conversation}''
\end{itemize}

\textbf{DEADLINE: Tuesday 19 August: Response Paper 2 (film analysis)}

\textbf{DEADLINE: Friday 22 August: Paper 2/Creative Project}

\begin{center}\rule{0.5\linewidth}{0.5pt}\end{center}

\phantomsection\label{refs}
\begin{CSLReferences}{1}{1}
\bibitem[\citeproctext]{ref-althusser1971ideology}
Althusser, Louis. 1971. {``Ideology and Ideological State Apparatuses
(Notes Towards an Investigation).''} In \emph{Lenin and Philosophy and
Other Essays}, translated by Ben Brewster. Monthly Review Press.

\bibitem[\citeproctext]{ref-barthes1972mythologies}
Barthes, Roland. 1972. \emph{Mythologies}. Translated by Annette Lavers.
Hill; Wang.

\bibitem[\citeproctext]{ref-baudrillard1994simulacra}
Baudrillard, Jean. 1994. \emph{Simulacra and Simulation}. Translated by
Sheila Faria Glaser. University of Michigan Press.

\bibitem[\citeproctext]{ref-baym2018music}
Baym, Nancy K., Daniel Cavicchi, and Norma Coates. 2018. {``Music Fandom
in the Digital Age: A Conversation.''} In \emph{The Routledge Companion
to Media Fandom}, edited by Melissa A. Click and Suzanne Scott.
Routledge.

\bibitem[\citeproctext]{ref-eco1986semiological}
Eco, Umberto. 1986. {``Towards a Semiological Guerrilla Warfare.''} In
\emph{Travels in Hyperreality: Essays}, translated by William Weaver.
Harcourt Brace Jovanovich.

\bibitem[\citeproctext]{ref-fisher2009capitalist}
Fisher, Mark. 2009. \emph{Capitalist Realism: Is There No Alternative?}
Zero Books.

\bibitem[\citeproctext]{ref-giolo2024aesthetics}
Giolo, Guilherme, and Michaël Berghman. 2024. {``The Aesthetics of the
Self: The Meaning-Making of Internet Aesthetics.''} \emph{Cultural
Sociology}, ahead of print.
\url{https://doi.org/10.1177/17499755241229570}.

\bibitem[\citeproctext]{ref-hall1980encoding}
Hall, Stuart. 1980. {``Encoding/Decoding.''} In \emph{Culture, Media,
Language: Working Papers in Cultural Studies, 1972-79}, edited by Stuart
Hall, Dorothy Hobson, Andrew Lowe, and Paul Willis. Hutchinson.

\bibitem[\citeproctext]{ref-jameson1991postmodernism}
Jameson, Fredric. 1991. \emph{Postmodernism, or, the Cultural Logic of
Late Capitalism}. Duke University Press.

\bibitem[\citeproctext]{ref-jenkins2012textual}
Jenkins, Henry. 2012. \emph{Textual Poachers: Television Fans and
Participatory Culture}. 2nd ed. Routledge.

\bibitem[\citeproctext]{ref-mulvey1975visual}
Mulvey, Laura. 1975. {``Visual Pleasure and Narrative Cinema.''}
\emph{Screen} 16 (3): 6--18.

\bibitem[\citeproctext]{ref-negra2007introduction}
Negra, Diane, and Yvonne Tasker. 2007. {``Introduction: Feminist
Politics and Postfeminist Culture.''} In \emph{Interrogating
Postfeminism: Gender and the Politics of Popular Culture}, edited by
Diane Negra and Yvonne Tasker. Duke University Press.

\bibitem[\citeproctext]{ref-richardson2020bearing}
Richardson, Allissa V. 2020. \emph{Bearing Witness While Black: African
Americans, Smartphones, and the New Protest \#Journalism}. Oxford
University Press.

\bibitem[\citeproctext]{ref-ulman2014excellences}
Ulman, Amalia. 2014. \emph{Excellences \& Perfections}. Instagram
performance.

\end{CSLReferences}




\end{document}
